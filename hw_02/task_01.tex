\ssr{Задание 1}
\noindent \textbf{Условие:} Пассажир оставил вещи в автоматической камере хранения, а когда при-
шел их получать, вспомнил лишь, что в кодовой последовательности $(x_1, x_2, x_3, x_4)$ была под-
последовательность $(2, 3)$. Какова вероятность того, что он с первой попытки наберет нужный
четырехзначный номер?

\section*{Решение:}

В контексте этой задачи под "подпоследовательностью" будем подразумевать, что цифры 2 и 3 идут подряд. То есть возможные позиции для пары $(2,3)$:

Позиции 1 и 2: $(2, 3, x_3, x_4)$

Позиции 2 и 3: $(x_1, 2, 3, x_4)$

Позиции 3 и 4: $(x_1, x_2, 2, 3)$

\vspace{10 pt}

Определим общее число благоприятных исходов
У нас есть 3 возможных положения для пары (2,3). Для каждого из этих случаев две оставшиеся позиции могут быть заняты любыми цифрами от 0 до 9.

Коды вида $2 3 \_ \_$: $10 \cdot 10$ = 100 кодов.

Коды вида $\_ 2 3 \_$: $10 \cdot 10$ = 100 кодов.

Коды вида $\_ \_ 2 3$: $10 \cdot 10$ = 100 кодов.

Проверим пересечения: «23» одновременно на $(1,2)$ и $(2,3)$ невозможно (противоречие), и на $(2,3)$ и $(3,4)$ тоже невозможно. Единственная возможная пара пересечения — (1,2) и (3,4): это последовательность: $(2, 3, 2, 3)$.

По формуле включений-исключений находим общее число благоприятных исходов: 100 + 100 + 100 - 1 = 299 кодов.

Тогда вероятность того, что пассажир с первой попытки наберет нужный четырехзначный номер равна:
$P = \frac{1}{299}$


\begin{center}
	\fbox{
		\parbox{\widthof{$\displaystyle \frac{1}{299}$} + 4em}{
			\centering
			\textbf{Ответ:} $\displaystyle \frac{1}{299}$
		}
	}
\end{center}

\clearpage