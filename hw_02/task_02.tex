\ssr{Задание 2}
\noindent \textbf{Условие:} Завод изготавливает изделия, каждое из которых с вероятностью $p = 0.01$ может иметь дефект. Каков должен быть объем случайной выборки, чтобы вероятность встретить в ней хотя бы одно дефектное изделие была не менее $0.95$?

\section*{Решение:}
\textbf{Введем обозначения:}
\begin{itemize}
	\item $p = 0.01$ – вероятность дефекта для одного изделия;
	\item $q = 1 - p = 0.99$ – вероятность того, что одно изделие годное;
	\item $n$ – объем выборки;
	\item событие $A$ – в выборке хотя бы одно дефектное изделие;
	\item событие $\bar{A}$ – все изделия в выборке без дефектов;
\end{itemize}

Поскольку выборка случайная, вероятность того, что все $n$ изделий годные, равна:
$P(\bar{A}) = (0.99)^n$

Вероятности событий $P(A)$ и $P(\bar{A})$ связаны следующим соотношением: $P(A) = 1 - P(\bar{A})$

\vspace{10pt}

\textbf{Составим неравенство}

Нам нужно, чтобы $P(A) \ge 0.95$. Подставим выражение для $P(A)$:

$1 - P(\bar{A}) \ge 0.95$

$1 - (0.99)^n \ge 0.95$

$-(0.99)^n \ge 0.95 - 1$

$-(0.99)^n \ge -0.05$

$(0.99)^n \le 0.05$

\vspace{10pt}

Прологарифмируем обе части:

$\ln((0.99)^n) \le \ln(0.05)$

$n \cdot \ln(0.99) \le \ln(0.05)$

\vspace{10pt}

Так как $\ln(0.99) < 0$, то

$n \ge \frac{\ln(0.05)}{\ln(0.99)} \approx 298.07$

Так как объем выборки должен быть целым, то $n = 299$

\begin{center}
	\fbox{
		\parbox{\widthof{$\displaystyle n = 299$} + 4em}{
			\centering
			\textbf{Ответ:} $\displaystyle n = 299$
		}
	}
\end{center}