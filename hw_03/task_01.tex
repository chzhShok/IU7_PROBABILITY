\ssr{Задание 1}
\noindent \textbf{Условие:} Прочность $X$ некоторого образца имеет нормальный закон распределения с математическим ожиданием $m_1 = 9 \ \text{МПа}$ и дисперсией $\sigma_1^2 = 1 \ \text{МПа}^2$. На образец действует случайная нагрузка $Y$, распределенная по нормальному закону с математическим ожиданием $m_2 = 4 \ \text{МПа}$ и дисперсией $\sigma_2^2 = 4 \ \text{МПа}^2$. Найти вероятность неразрушения образца, то есть вероятность события $\{X >Y\}$.

\section*{Решение:}

По условию: $X = \mathcal{N}(9, 1), Y = \mathcal{N}(4, 4)$

\textbf{Переход к разности \(Z = X - Y\)}

Так как \(X\) и \(Y\) независимы и нормальны, разность $Z = X - Y$ также распределена нормально.

Математическое ожидание:
\[
m_Z = m_1 - m_2 = 9 - 4 = 5 \ \text{МПа}.
\]

Дисперсия:
\[
\sigma_Z^2 = \sigma_1^2 + \sigma_2^2 = 1 + 4 = 5 \ \text{МПа}^2.
\]

Таким образом,
\[
Z \sim \mathcal{N}(5, 5).
\]

Событие $\{X > Y\}$ равносильно $\{Z > 0\}$.

\vspace{10pt}

\textbf{Стандартизация}

Введём стандартную нормальную величину:
\[
U = \frac{Z - m_Z}{\sigma_Z} = \frac{Z - 5}{\sqrt{5}},
\]
\[
U \sim \mathcal{N}(0, 1).
\]

Тогда
\[
P(Z > 0) = P\left(U > \frac{0 - 5}{\sqrt{5}}\right) = P\left(U > -\sqrt{5}\right).
\]

\vspace{10pt}

\textbf{Вычисление вероятности}

Из симметрии стандартного нормального распределения:
\[
P(U > -a) = P(U < a) = \Phi(a),
\]
где \(\Phi\) --- функция стандартного нормального распределения.

Здесь \(a = \sqrt{5} \approx 2.236\).

По таблицам функции \(\Phi(x)\):
\begin{align*}
	\Phi(2.23) &\approx 0.9871, \\
	\Phi(2.24) &\approx 0.9875.
\end{align*}

Интерполяция для \(x = 2.236\):
\[
\Phi(2.236) \approx 0.9871 + 0.6 \cdot (0.9875 - 0.9871) 
= 0.9871 + 0.00024 = 0.98734.
\]

Таким образом,
\[
P(Z > 0) = \Phi(\sqrt{5}) \approx 0.98734.
\]

\begin{center}
	\fbox{
		\parbox{\widthof{$\displaystyle P(X>Y) \approx 0.98734$} + 4em}{
			\centering
			\textbf{Ответ:} $\displaystyle P(X>Y) \approx 0.98734$
		}
	}
\end{center}

\clearpage