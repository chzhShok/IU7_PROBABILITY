\ssr{Задание 3}
\noindent \textbf{Условие:} Найти объем тела, ограниченного поверхностями $y=0, \, z=0, \, x+y+z=4, \, 2x+z=4$

\section*{Решение:}

Дано:
\begin{itemize}
	\item $y=0$ – координатная плоскость $xOz$;
	\item $z=0$ – координатная плоскость $xOy$;
	\item $x+y+z=4$ – плоскость, отсекающая на осях координат отрезки $x=4$, $y=4$, $z=4$;
	\item $2x+z=4$ – плоскость, параллельная оси $Oy$.
\end{itemize}

\begin{figure}[H]\label{task_03_01} 
	\centering
	\includegraphics[width=0.8\textwidth]{./images/task\_03\_01.png}
	\caption{Область тела, ограниченная $y=0, \, z=0, \, x+y+z=4, \, 2x+z=4$}
	\label{fig:image_03_01}
\end{figure}

Объем тела будем вычислять как сумму двух объемов, поскольку тело состоит из двух частей, разделенных линией пересечения плоскостей $x+y+z=4$ и $2x+z=4$.

Найдем линию пересечения этих плоскостей:
\[
\left\{
\begin{array}{rcl}
	x+y+z & = & 4 \\
	2x+z & = & 4
\end{array}
\right.
\Rightarrow y = x
\]

Таким образом, тело разделяется на две части плоскостью $y=x$.

Сначала вычислим объем для фигуры, где $x \in [2,4]$.

В этой области верхней поверхностью является плоскость $z = 4-x-y$, а нижней – плоскость $z=0$.

Найдем границы области $D_1$ – проекции на плоскость $xOy$(рис. \ref{fig:image_03_02}):
\begin{itemize}
	\item Пересечение плоскости $x+y+z=4$ с плоскостью $z=0$: $y = 4-x$
	\item Граница $y=0$ (координатная плоскость)
	\item Граница $x=2$ (из пересечения с другой частью тела)
	\item Граница $x=4$ (из пересечения с плоскостью $x+y+z=4$ при $y=0$, $z=0$)
\end{itemize}

\begin{figure}[H]\label{task_03_02} 
	\centering
	\includegraphics[width=0.8\textwidth]{./images/task\_03\_02.jpg}
	\caption{Проекция на плоскость $xOy$ при $x \in [2, 4]$}
	\label{fig:image_03_02}
\end{figure}

Таким образом:
\[
V_1 = \iint\limits_{D_1} z\,dx\,dy = \int_{2}^{4} dx \int_{0}^{4-x} (4-x-y)\,dy
\]

Вычисляем внутренний интеграл по $y$:
\[
\int_{0}^{4-x} (4-x-y)\,dy = \left[4y - xy - \frac{y^2}{2}\right]\Biggr|_{0}^{4-x} = 4(4-x) - x(4-x) - \frac{(4-x)^2}{2}
\]

Упрощаем выражение:
\[
4(4-x) - x(4-x) - \frac{(4-x)^2}{2} 
\]
\[
= 16 - 4x - 4x + x^2 - \frac{(16 - 8x + x^2)}{2} = 16 - 8x + x^2 - 8 + 4x - \frac{x^2}{2}
\]
\[
= 8 - 4x + \frac{x^2}{2}
\]

Теперь вычисляем внешний интеграл по $x$:
\[
V_1 = \int_{2}^{4} \left(8 - 4x + \frac{x^2}{2}\right) dx = \left[8x - 2x^2 + \frac{x^3}{6}\right]\Biggr|_{2}^{4}
\]

Подставляем пределы:
\[
\left(32 - 32 + \frac{64}{6}\right) - \left(16 - 8 + \frac{8}{6}\right) = \frac{64}{6} - \left(8 + \frac{8}{6}\right) = \frac{64}{6} - \frac{56}{6} = \frac{8}{6} = \frac{4}{3}
\]

\vspace{10pt}

Теперь вычислим объем для фигуры, где $x \in [0,2]$.

В этой области тело ограничено сверху плоскостью $z = 4-x-y$, а снизу – плоскостью $z = 4-2x$.

Найдем границы области $D_2$ – проекции на плоскость $xOy$(рис. \ref{fig:image_03_03}):
\begin{itemize}
	\item Граница $y=0$ (координатная плоскость)
	\item Граница $y=x$ (линия пересечения плоскостей)
	\item Граница $x=0$ (координатная плоскость)
	\item Граница $x=2$ (из пересечения с первой частью тела)
\end{itemize}

\begin{figure}[H]\label{task_03_03} 
	\centering
	\includegraphics[width=0.8\textwidth]{./images/task\_03\_03.jpg}
	\caption{Проекция на плоскость $xOy$ при $x \in [0, 2]$}
	\label{fig:image_03_03}
\end{figure}

Таким образом:
\[
V_2 = \iint\limits_{D_2} (z_{\text{верх}} - z_{\text{нижн}})\,dx\,dy = \int_{0}^{2} dx \int_{0}^{x} [(4-x-y) - (4-2x)]\,dy
\]

Упрощаем подынтегральное выражение:
\[
(4-x-y) - (4-2x) = x - y
\]

Следовательно:
\[
V_2 = \int_{0}^{2} dx \int_{0}^{x} (x - y)\,dy
\]

Вычисляем внутренний интеграл по $y$:
\[
\int_{0}^{x} (x - y)\,dy = \left[xy - \frac{y^2}{2}\right]\Biggr|_{0}^{x} = x^2 - \frac{x^2}{2} = \frac{x^2}{2}
\]

Теперь вычисляем внешний интеграл по $x$:
\[
V_2 = \int_{0}^{2} \frac{x^2}{2} dx = \frac{1}{2} \cdot \frac{x^3}{3} \Biggr|_{0}^{2} = \frac{1}{6} \cdot 8 = \frac{4}{3}
\]

\vspace{10pt}

Суммируем объемы обеих частей:
\[
V = V_1 + V_2 = \frac{4}{3} + \frac{4}{3} = \frac{8}{3}
\]

\begin{center}
	\fbox{
		\parbox{\widthof{$\displaystyle \frac{8}{3}$} + 4em}{
			\centering
			\textbf{Ответ:} $\displaystyle \frac{8}{3}$
		}
	}
\end{center}

\clearpage
